\chapter{Wavelength Calibration} \label{chapter:Wavelength Calibration}
\begin{enumerate}
\item Why do we need to calibrate wavelength?
\begin{itemize}
\item The spectrograph's detector can only record intensity (\textit{I}) as a function of pixel location 
\item We need some method to transition from this I(x,y) $\rightarrow I(\lambda$)
\item A Wavelength Solution converts (x,y) $\rightarrow \lambda : \lambda_{soln} = f(x,y)$
\item In the EXPRES instrument's case, this x corresponds to the calculated One-Dimensional pixel value of a given Echelle Order and y corresponds to the Echelle order itself
\end{itemize}

\item How \textit{precise} do we need to be?
\begin{itemize}
\item $\frac{\Delta \lambda}{\lambda} \approx \frac{\Delta v}{v}$
\item A change in wavelength of 1 \AA\ at 5000 \AA\ equates to an enormous RV of m s \textsuperscript{-1}
\end{itemize}
\item In order to convert from $(x,y) \rightarrow \lambda$ we need a series of fixed reference wavelengths.
\item EXPRES' Wavelength Calibration Sources

\begin{itemize}
\item \textbf{Thorium Argon} cathode lamp (\textit{ThAr})  produces atomic spectral lines
\begin{itemize}
\item Current catalog gathered from NIST, the lines are not specific to EXPRES
\item Wide wavelength range from between 3500 and 8500 \AA\
\item Used to generate a course wavelength solution $\lambda_{ThAr}$(order, pixel) which then assigns mode number to the LFC lines
\end{itemize}
\item \textbf{Laser Frequency Comb} (\textit{LFC}) produces a "picket fence" of sharp lines at even spacing within frequency space
\begin{itemize}
\item If the mode number of a line is known (from the ThAr solution), its central wavelength can be very precisely characterized
\item Produces an abundance of lines, one every $\sim$ 10 pixels
\end{itemize}

\end{itemize}
\item -
\end{enumerate}
